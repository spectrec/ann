\documentclass[12pt]{article}

\usepackage[utf8]{inputenc}
\usepackage[english, russian]{babel}

\RequirePackage[left=30mm,right=20mm,top=20mm,bottom=20mm,headsep=0pt]{geometry}

\begin{document}

\paragraph{Приветствие (слайд 1).}
Здравствуйте, уважаемые члены комиссии.

\paragraph{Цели и задачи работы (слайд 2).}
Моя работа направлена на разработку учебной операционной системы, которая может
быть использована в курсе <<Проектирование операционных систем>>.

Для выполнения работы необходимо было выполнить следующие задачи:
\begin{enumerate}
\item Провести анализ архитектуры AMD64 и существующих учебных ОС;
\item Разработать структуру ОС;
\item Разработать загрузчик ОС;
\item Разработать модули ОС;
\item Определить последовательность выполнения лабораторных работ.
\end{enumerate}

Следует также отметить что ОС является учебной, поэтому дополнительным требованием
является возможность работать с постепенно усложняющимся кодом ОС.

\paragraph{Допущения и ограничения (слайд 3).}
На слайде 3 представлены ограничения и допущения, сделанные в работе: для ОС необходимо
не менее 32 мегабайт доступной физической памяти и не менее 40 мегабайт дискового пространства.

В разработанной ОС отсутствует поддержка файловой системы и многопроцессорной обработки.
Для упрощения реализации, код ядра выполняется с отключенными прерываниями.

Для запуска ОС необходимо использовать эмулятор QEMU.

\paragraph{Структура ОС (слайд 4).}
На слайде 4 представлена структура ОС. Основными компонентами являются загрузчик и ядро.
Рассмотрим их более подробно.

\paragraph{Загрузчик ОС (слайд 5).}
Для загрузки ОС необходимо выполнить следующие действия: активировать линию А20,
определить доступные области физической памяти, загрузить ядро, перейти в длинный
режим, подготовить требуемые ядру структуры данных, передать управление ядру.

Все эти действия не получится вместить в 510 байт, поэтому в разработанной ОС используется
двухэтапная загрузка. BIOS загружает с диска первый сектор (512 байт), в котором находится
первый загрузчик и передает ему управление.

Первый загрузчик активирует линию А20, определяет доступные физические области, используя
прерывания BIOS, переходит в защищенный режим, загружает и передает управление второму
загрузчику.

Второй загрузчик определяет доступный объем физической памяти, используя информацию, полученную
первым загрузчиком, подготавливает требуемые для работы ядра отображения и структуры данных:
GDT, PML4, массив дескрипторов страниц. Переходит в длинный режим, загружает и передает
управление ядру.

\paragraph{Действия для перехода в длинный режим (слайд 6).}
Для перехода в длинный режим второй загрузчик должен активировать PAE (CR4.PAE=1), создать
дескрипторы сегментов кода для 64-битного режима, активировать длинный режим, EFER.LME=1,
загрузить адрес PML4 в регистр CR3 и активироваться страничное преобразование (CR0.PG=1).

\paragraph{Управление памятью (слайд 7).}
Одной из задач ОС является управление памятью. В длинном режим механизм сегментного преобразования
отключен и используется только страничное преобразование памяти. Для этого используется 4-уровневая
иерархия таблиц страниц. Виртуальный адрес при этом делится на 6 частей: 4 из которых являются
индексами в таблицах страниц.

Механизм страничного преобразования используется чтобы обеспечить каждый процесс областью
физической памяти для хранения кода и данных, недоступной другим процессам. Процесс не может
получить доступ к физической памяти, которая не отображена в его адресное пространство системным ПО.

Для организации отображения виртуальных адресов в определенные физические необходимо заполнить
таблицы страниц.

\paragraph{Схема алгоритма создания отображения (слайд 8).}
На слайде 8 показана схема алгоритма создания отображения виртуального адреса в физический.

\paragraph{Схема алгоритма поиска элемента в таблице страниц (слайд 9).}
При создании отображения необходимо выполнять поиск элемента, указывающего на следующий элемент
в иерархии таблиц страниц. На слайде 9 показана схема алгоритма поиска элемента в таблице страниц.

\paragraph{Контекст процесса (слайд 10).}
Еще одной задачей ОС является управление процессами. Для реализации многозадачности с
использованием одного ядра ЦП необходимо иметь возможность приостанавливать и продолжать процессы.

Для этого необходимо хранить контекст процесса (значения регистров и адрес PML4). На данном слайде
показан формат дескриптора процесса. Формат выбран таким образом, чтобы он заполнялся естественным
образом при возникновении прерываний.

\paragraph{Дескриптор процесса (слайд 11).}
Однако, для управления процессами одного контекста процесса недостаточно, поэтому каждый
процесс имеет дескриптор процесса, который включает: контекст процесса, имя и идентификатор процесса,
текущее состояние, и виртуальный адрес PML4. При работе с процессами ОС оперирует
именно дескрипторами процессов.

\paragraph{Состояние процессов (слайд 12).}
Помимо контекста и дескриптора, каждый процесс имеет состояние. В разработанной ОС процесс
может находиться в одном из четырех состояний: FREE, READY, RUN, DONT\_RUN. На слайде 11
показана диаграмма переходов состояний процессов.

\paragraph{Системные вызовы (слайд 13).}
Для работы прикладных программ, как правило, необходим доступ к различным сервисам ядра: вывод
на экран, выделение памяти, создание процессов.

Однако, давать прикладным процессам прямой доступ к данным возможностям не безопасно, т.к.
ошибка в прикладном процессе может привести к полной неработоспособности всего ядра.

Поэтому для обращения к системным сервисам ядро предоставляет строго определенные точки входа:
обработчики системных вызовов. Процесс передачи управление такому обработчику называется системным вызовом.

В разработанной ОС реализованы 4 системных вызова: PUTS - выводит на экран строку, завершающуюся нулевым
символом. YIELD - вызывает планировщик, для передачи управления следующему процессу в очереди. EXIT -
уничтожает процесс, FORK - создает копию процесса.

\paragraph{Копирование при записи (слайд 14).}
В разработанной ОС для создания процессов используется системный вызов FORK. При выполнении системного
вызова создается новый процесс (процесс-потомок).

Процесс потомок полностью идентичен процессу-родителю, за исключением одного момента: системный вызов
возвращает 0 в процессе-потомке и идентификатор процесса (отличное от 0 значения) в процессе-родителе. Для
клонирования процессов используется механизм копирования при записи. Все доступные для записи
страницы помечаются как доступные только для чтения, т.е. для них сбрасывается бит <<W>> и устанавливается бит <<COW>>.

При попытке записи одним из процессов в страницу доступную только для чтения произойдет страничное исключение.
Обработчик страничного исключения, для страниц у которых установлен бит COW, выделяет новую страницу из списка
свободных страниц, копирует в нее содержимое оригинальной страницы и отображает новую страницу вместо старой.

Данный процесс схематично показан на слайде 14.

\paragraph{Тестирование и отладка (слайд 15).}
Для тестирования ядра ОС написаны прикладные программы (функциональные тесты), которые позволяют
проверить функции вывода данных на экран; ограничения на чтение и запись данных в область ядра
и неотображенные области; функции создания и уничтожения процессов; и работу вытесняющей многозадачности.

Для отладки использовался стандартный для unix-подобных систем отладчик GDB.

\paragraph{Последовательность проведения лабораторных работ.}
А теперь я расскажу о том, как можно использовать данную ОС при проведении лабораторных работ.
Предполагается что ОС будет разделена на несколько модулей и в каждой лабораторной работе необходимо
будет изучить или модифицировать один или несколько модулей.

Итак, лабораторная №1. Изучается как происходит загрузка компьютера, т.е. все начинается с загрузки
и передачи управления BIOS'ом первому загрузчику. В данной работе необходимо изучить устройство GDT
и процесс перехода в защищенный режим.

Лабораторная №2. После этого можно переходить ко второму загрузчику, который должен подготовить данные
для загрузки ОС. Здесь студенты знакомятся с механизмом работы страничного преобразования для загрузки
ядра в оперативную память и процессом перехода в длинный режим.

Лабораторная №3. После перехода в длинный режим управление передается ядру ОС, которое продолжает
инициализацию. Предлагается начать с настройки механизма обработки прерываний: здесь студенты познакомятся
с таблицей IDT, новым механизмом переключения стека -- IST и узнают как настраивать APIC и IOAPIC.

Лабораторная №4. После настройки прерываний предлагается заняться управлением процессами. В данной работе
будет введено понятие контекста и дескриптора процесса. Реализованы функции создания, уничтожения и переключения
процессов.

Лабораторная №5. В конце предлагается реализовать механизм обработки системных вызовов,
используя механизм обработки прерываний. В этой же работе необходимо реализовать механизм
копирования при записи.

\paragraph{Выводы (слайд 16).}
Итак, в результате выполнения дипломной работы, была разработана учебная операционная
система под современную архитектуру AMD64, обладающая следующими особенностями:
двухэтапная загрузка, страничное управление памятью, возможность исполнения прикладных
программ и запуска потоков ядра, поддержка вытесняющей многозадачности и поддержка
механизма копирования при записи.

Доклад окончен. Готов ответить на ваши вопросы.

\end{document}
