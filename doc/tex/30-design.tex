\chapter{Конструкторский раздел}
\label{cha:design}

%Исходя из аналита (и ссылаясь на), ты говоришь, какой режим выбираешь. Дальше раздел 2.7 туда едет (про переход), только лишнее надо убрать (инициализируешь ли ты IDT и страничное преобразование в загрузчике? LDT?). Ну и потом рассказываешь, как из всего того, о чем ты рассказал в аналите, собрать работающее ядро ОС. В картинках схемах и диаграммах.


\section{Загрузчик ОС}
Для запуска операционной системы необходимо настроить минимальное окружение (активировать линию A20,
загрузить GDT, перейти в защищенный режим), загрузить ос в память и передать ей управление.
Этим занимается загрузчик операционной системы (англ. bootloader). Существует множество загрузчиков
(GRUB, Lilo, Gujin, Bootf и др.), однако они не лучшим образом подходят для учебных целей, т.к. поддерживают
различное оборудование и файловые системы (что приводит к увеличению объемов и усложнению исходного кода),
поэтому было принято решение реализовать собственный загрузчик.


\section{Инициализация процессора и переход в длинный режим}
\label{sec:long_mode_activation}

\subsection{Инициализация реального режима}
Перед переходом в защищенный режим необходимо настроить базое окружение
реального режима. Это окружение включает:
\begin{itemize}
\item IDT реального режима, содержащую адреса точек входа в обработчики прерываний
	и исключений реального режима. После инициализации процессора базовый адрес IDT
	равен 0. При необходимости системное ПО может изменить его, загрузив новый адрес
	в регистр IDTR.
\item Обработчики прерываний и исключений реального режима. Должны быть загружены до
	включения внешних прерываний.
\item Указатель стека (SS:SP). Можно использовать значения SS:SP проинициализированные процессором.
\item Минимум один селектор сегмента данных для хранения структур данных защищенного режима,
	созданных в реальном режиме.
\end{itemize}

После того как выполнена инициализация реального режима, ПО может переходить к
инициализации защищенного режима.

\subsection{Инициализация защищенного режима}
Перед активацией длинного режима необходимо перейти в защищенный режим.
Рабочее окружение защищенного режима включает:
\begin{itemize}
\item IDT защищенного режима.
\item Обработчики прерываний и исключений, на которые ссылается IDT.
\item GDT, которая содержит:
	\begin{itemize}
	\item Дескриптор сегмента кода для защищенного режима.
	\item Доступный для чтения/записи сегмент данных, который можно использовать как стек защищенного режима.
	\end{itemize}
\end{itemize}

Системное ПО при необходимости может загрузить GDT, содержащую несколько дескрипторов сегментов данных,
дескрипторы TSS и LDT для ПО, которое выполняет инициализацию длинного режима.

После того, как структуры данных защищенного режима были проинициализированы, системному ПО необходимо
загрузить в регистры IDTR и GDTR (а также, при необходимости -- в LDTR и TR) указатели на эти структуры данных.
После того, как эти регистры загружены, переход в защищенный режим можно выполнить установив CR0.PE в 1.

Если при инициализации длинного режима используется унаследованный механизм страничного преобразования --
сначала необходимо проинициализировать таблицы страниц. Необходим как минимум один каталог страниц и одна
таблица страниц. Кроме того, необходимо загрузить в регистр CR3 физический адрес таблицы страниц верхнего
уровня. После завершения инициализации этих структур данных и перехода в защищенный режим, можно
включить страничное преобразование, установив CR0.PG в 1.

\subsection{Инициализация длинного режима}
В защищенном режиме системное ПО может подготовить структуры данных, необходимые для перехода в
длинный режим и сохранить их в произвольном месте в пределах первых 4х гигабайт физической памяти.
Эти структуры данных можно будет переместить за пределы 4х гигабайт после перехода в длинный режим.
Для перехода в длинный режим необходимы следующие структуры данных:
\begin{itemize}
\item IDT, содержащая 64-битные дескрипторы шлюзы прерываний.
\item 64-битные обработчики прерываний и исключений.
\item GDT, содержащая дескрипторы сегментов для ПО работающего в 64-битном режиме и в
	режиме совместимости, включающая:
	\begin{itemize}
	\item Десприпторы LDT, необходимые ОС и/или прикладному ПО.
	\item Дескрипторы TSS (минимум один).
	\item Дескрипторы сегментов кода для переключения между 64-битным режимом и режимом совместимости.
	\item Дескрипторы сегментов данных для ПО, работающего в режиме совместимости.
	\end{itemize}

	Существующая GDT защищенного режима может быть использована для хранения перечисленных
	выше дескрипторов.
\item 64-битный TSS для хранения указателей стека для 0, 1 и 2го уровней привилегий и/или указателей IST.
\item 4 уровня таблиц страниц. Для перехода в длинный режим, также необходимо активировать PAE.
\end{itemize}

Если страничное преобразование было активировано в процессе инициализации, его необходимо деактивировать перед
активацией длинного режима. После подготовки структур данных длинного режима и деактивации страничного преобразования,
можно выполнить активацию и переход в длинный режим.

\subsection{Активация и переход в длинный режим}
Для активации длинного режима необходимо установить бит EFER.LME в 1. Однако переход в длинный режим
не будет выполнен до активациии страничного преобразования. После того, как системное ПО
активирует страничное преобразование, при активированном длинном режиме, процессор перейдет в
длинный режим, установив при этом бит EFER.LMA в 1.

Для выбора подрежима работы процессора в длинном режиме используются 2 бита дескриптора сегмента кода (CS.L и CS.D).
Комбинация CS.L=1, CS.D=0 -- переводит процессор в 64-битный режим. Комбинация CS.L=0, CS.D=0 -- в
режим совместимости. Комбинация CS.L=1, CS.D=1 зарезервирована для будущего использования.

\subsubsection*{Переход в длинный режим}
Переход в длинный режим выполняется в несколько этапов: деактивация страничного преобразования (CR0.PG=0),
активация механизма расширения физических адресов (CR4.PAE=1), загрузка регистра CR3,
активация длинного режима (EFER.LME=1), активация страничного преобразования (CR0.PG=1).

Таким образом, для перехода в защищенный режим, системное ПО должно выполнить следующие действия:
\begin{enumerate}[1.]
\item Если используется страничное преобразование, его необходимо деактивировать, сбросив бит CR0.PG в 0.
	Для этого необходимо, чтобы инструкция MOV CR0, используемая для деактивации страничного преобразования
	располагалась на странице, виртуальный адрес которой совпадает с физическим.
\item В любой последовательности:
	\begin{itemize}
	\item Активировать механизм расширения физических адресов, установив бит CR4.PAE в 1. Это
		необходимо сделать до активации страничного преобразования.
	\item Загрузить в регистр CR3 физический адрес PML4.
	\item Активировать длинный режим, установив бит EFER.LME в 1.
	\end{itemize}
\item Активировать страничное преобразование, установив бит CR0.PG в 1. В результате процессор установит бит EFER.LMA в 1.
\end{enumerate}

\subsubsection*{Обновление ссылок на таблицы системных дескрипторов}
После перехода в длинный режим регистры таблиц дескрипторов (GDTR, LDTR, IDTR, TR) ссылаются на таблицы унаследованного
режима, которые расположены в пределах первых 4-х гагабайт виртуального адресного пространства. Системному ПО необходимо
обновить эти регистры, чтобы они указывали на 64-битные версии таблиц дескрипторов, используя команды LGDT, LIDT, LLDT, LTR.

В длинном режиме необходимо использовать 64-битные дескрипторы шлюзов прерываний. Прерывания должны быть
отключены пока IDTR не будет обновлена. Для отключения внешних прерываний можно воспользоваться инструкцией CLI.

\subsubsection*{Перемещение таблиц страничного преобразования}
После перехода в длинный режим таблицы страниц расположены в пределах первых 4-х гигабайт физического
адресного пространства, т.к. инструкция MOV CR3 выполняется в защищенном режиме (в регистр CR3 записываются
только младшие 32 бита физического адреса). После перехода в длинный режим
системное ПО может переместить таблицы страниц в произвольное место физического адресного пространства,
обновив значение регистра CR3.

