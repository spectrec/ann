\Introduction

Для изучения основ функционирования операционных систем недостаточно
изучения одного теоретического материала: для понимания работы ядра
ОС нужно изучить и его исходные коды.

Хотя в настоящее время ядра многих эксплуатируемых ОС доступны для
такого изучения, все они как возможный методический материал для
учебного процесса обладают двумя недостатками: высокой сложностью
и большим объемом исходных текстов. По этой причине коллективом
сотрудников университета MIT была разработана учебная операционная
система, позже названная JOS, предназначенная для практической
части курса разработки операционных систем.

Помимо своей компактности, JOS как учебная система обладает ещё одним
достоинством: студент работает со все более усложняющимися вариантами
системы, постепенно знакомясь с теми или иными функциями ядра ОС.
Исходные тексты JOS имеют лицензию MIT или BSD, т.е. она является
свободным программным обеспечением и может использоваться с минимальными
ограничениями.

Однако JOS имеет один существенный недостаток: она разработана для
устаревшей архитектуры x86.

Таким образом разработка учебной операционной системы для более современной
архитекруты AMD64 является актуальной задачей.
