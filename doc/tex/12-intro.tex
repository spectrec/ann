\Introduction

Для изучения основ функционирования операционных систем недостаточно
изучения только теоретического материала. Для понимания работы ядра
ОС необходимо изучать и модифицировать его исходный код.

В настоящее время существует множество операционных систем с открытым
исходным кодом: GNU Linux, FreeBSD, ReactOS и др. Однако, ядра этих
операционных систем плохо подходят для учебного процесса, т.к. они
имеют большой объем исходного кода и обладают высокой сложностью.

По этой причине были разработаны несколько учебных операционных
систем: JOS, xv6 и PhantomEx. Ядра этих ОС имеют небольшой объем
исходного кода и небольшую сложность, по сравнению с
эксплуатируемыми ОС, что делает их пригодными для обучения.

Однако, данные ОС имеют один существенный недостаток: они разработаны
под устаревшую архитектуру x86.

Поэтому было принято решение разработать учебную ОС под более современную
архитектуру x86\_64, автор решил назвать ее \textbf{Ann}.

Данное пособие предназначено для проведения лабораторных работ по курсу
разработки операционных систем и построено следующим образом. Первая глава
пособия содержит краткие сведения о работе процессоров семейства x86\_64,
организации в них виртуальной памяти и обработки прерываний. Вторая глава
посвящена обзору исходных текстов Ann и используемых диалектов языков
программирования. Остальные главы содержат описание практических занятий по
разработке операционной системы на базе имеющихся исходных кодов Ann.

В конце глав приводятся контрольные вопросы для проверки усвоенных студентом
знаний. Некоторые главы содержат задания по написаню исходного кода, которого
не хватает для полноценной работы разрабатываемой операционной системы.

Для выполнения практических заданий пособия читателю потребуется подключенный
к интернету компьютер с POSIX-системой, имеющей следующие программы (указаны
версии, которые были у автора, более новые тоже должны подойти):
gcc-4.9.3, binutils-2.25.1, automake-1.15, autoconf-2.69, libtool-2.4.6,
perl-5.10, make-4.1, gdb-7.10.1, git-2.10.2, qemu-2.7.0,
произвольный текстовый редактор для работы с исходными текстами Ann (если не
знаете что выбрать - попробуйте vim).

Автор хотел бы поблагодарить за помощь Келарева Ивана Андреевича, Горина Сергея
Викторовича и Арышеву Анну Григорьевну.
